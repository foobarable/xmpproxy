\documentclass[notes=hide,yellow]{beamer}

% (c) 2008 Steffen Klemer <moh AT gmx BEEP org>
% This work is licensed under the Creative Commons Attribution-Share Alike 3.0
% Germany License. To view a copy of this license, visit
% http://creativecommons.org/licenses/by-sa/3.0/de/ or send a letter to Creative
% Commons, 171 Second Street, Suite 300, San Francisco, California, 94105, USA.
%
% See http://www.noch-mehr-davon.de/vortr.shtml
% Permissions beyond the scope of this license may be available at the same site
%
% Template based on: Copyright 2004 by Till Tantau <tantau@users.sourceforge.net>.


\mode<presentation>
{
%	\usetheme{AnnArbor} %Szeged
%	\usetheme{Berkeley}
	\usetheme{Frankfurt}
%	\usecolortheme{rose} %oder beaver oder rose oder orchid, albatross, rose
% 	\useinnertheme{circles}
%	\useoutertheme{split}
%	\setbeamercovered{invisible} %or transparent
% 	\usefottheme{professionalfonts}
% 	\usefonttheme[onlymath]{serif}
        %\setbeamercovered{invisible}
%	\setbeamertemplate{navigation symbols}{}
}

\usepackage{amsmath,amssymb,latexsym}
\usepackage{fancyvrb}
\usepackage{graphicx}
\usepackage{epstopdf}
\usepackage{amsfonts}
\usepackage{amsthm}
\usepackage{wasysym}
\usepackage{ucs}
\usepackage{listings}
\usepackage{stmaryrd}
\usepackage{hyperref}
\usepackage{graphics}
\usepackage{colortbl}

\usepackage{tikz}
\tikzstyle{every picture}+=[remember picture]
\usetikzlibrary{arrows}
\usetikzlibrary{shadows}
\usetikzlibrary{fit}
\usetikzlibrary{shapes}
\usetikzlibrary{backgrounds}

\tikzstyle{vertex}=[circle,fill=black!25,minimum size=12pt,inner sep=0pt]
\tikzstyle{selected vertex} = [vertex, fill=red!24]
\tikzstyle{blue selected vertex} = [vertex, fill=blue!25]
\tikzstyle{edge} = [draw,thick,-]
\tikzstyle{weight} = [font=\small]
\tikzstyle{selected edge} = [draw,line width=5pt,-,red!50]
\tikzstyle{ignored edge} = [draw,line width=5pt,-,black!20]
\tikzstyle{small vertex}=[circle,fill=black!25,minimum size=8pt, inner sep=0pt]
\tikzstyle{small selected vertex}=[circle,fill=red!25,minimum size=8pt, inner sep=0pt]



%\usepackage[ngerman]{babel}
%\usepackage[utf8x]{inputenc}




\title{ Distributed Filesystems}
\subtitle{ }
\author{Ralph Krimmel}



\begin{document}
	\begin{frame}
		\titlepage
	\end{frame}

	\begin{frame}
		\tableofcontents
	\end{frame}

\section{Definition}
\subsection*{}
\begin{frame}
\frametitle{Definition}
	\begin{block}{Definition}
		In computing, a distributed file system or network file system is any file system that allows access to files from multiple hosts via a computer network.$[1]$
	\end{block}
\end{frame}

\begin{frame}
\frametitle{Nothing new...}

		E.g. NFS developed by Sun Microsystems in 1984, but for NFS $<$ Version 4:
		
		\begin{itemize}
			\item Bad performance for huge number of clients 
			\item Hardly scalable
			\item Not fault tolerant
		\end{itemize}
\end{frame}

\begin{frame}
	\frametitle{Requirements for modern distributed filesystems}
	\begin{block}{Requirements}
	\begin{itemize}
		\item Performance
		\item Scalability
		\item Reliability
		\item Availability
	\end{itemize}
	\end{block}
\end{frame}

\begin{frame}
	\frametitle{Modern DFS}
	There are several approaches reaching this requirements. \\
	We will have a look on how Google took care for those in GFS	
\end{frame}

\section{GFS}
\subsection*{}
\begin{frame}
	\frametitle{Short introduction into GFS - Cluster}
	\begin{block}{GFS cluster}
		\begin{itemize}
			\item A single master server
			\item Multiple chunkservers
		\end{itemize}
	\end{block}
\end{frame}

\begin{frame}
	\frametitle{Short introduction into GFS - Chunks}
	\begin{block}{Chunk}
		\begin{itemize}
			
			\item Files are divided into chunks %fixed size
			\item Identified by an 64 bit chunk handle
			\item Chunks are stored as files on linux filesystems of the chunkservers
		\end{itemize}
	\end{block}
	
\end{frame}

\begin{frame}
	\frametitle{Short introduction into GFS - Master server}
	The master server stores and manages all the metadata
	\begin{block}{Master}
	\begin{itemize}
		\item Namespace (/foo/bar/.../)
		\item Access Control
		\item File $\rightarrow$ Chunk Mapping
		\item Location of chunks
	\end{itemize}
	\end{block}
	All metadata is stored in memory, some is synced to disc
\end{frame}

\begin{frame}
	\frametitle{Short introduction into GFS - Overview}
	\includegraphics[scale=0.45]{gfs_overview.png}
\end{frame}


\begin{frame}
	\frametitle{Short introduction into GFS - Master server}
	Additional tasks for the masterserver:	
	\begin{block}{Master}
	\begin{itemize}
		%raussuchen
		\item Chunk lease managment %mutation order, primary chunk, and so on..
		\item Garbage collection
		\item Chunk migration
	\end{itemize}
	\end{block}
\end{frame}

\begin{frame}
	\frametitle{Short introduction into GFS - Operations}
	\begin{itemize}
		\item Not POSIX compatible
		\item Standard operations: create, delete, close, read, write
		\item Snapshot %copy of file or directory cost for low cost
		\item Record append
	\end{itemize}
\end{frame}

\section{Reliability}
\subsection*{}
\begin{frame}
	\frametitle{Reliability}
	Problems: Component failures are the norm
	
	\begin{itemize}
		\item System consists of inexpensive hardware 	
		\item Application bugs %theres software that always crashes 
		\item Human failure % what was this cable for?
		\item Operating system bugs % ever used windows?
		\item Failures of 
		\begin{itemize}
			\item Harddisks
			\item Memory
			\item Connectors
			\item Networking
			\item ...
		\end{itemize}
	\end{itemize}		
	
\end{frame}

\begin{frame}
	\frametitle{Reliability in GFS}
	
	\begin{itemize}
		\item Each chunk is replicated on multiple chunkservers
		\item Master polls chunkservers periodically for state 	
		\item Re-Replication
		\item Chunk placement
	\end{itemize}

\end{frame}




\section{Performance}
\subsection*{}
\begin{frame}
	\frametitle{Performance}
	Observe the workload
	\begin{itemize}
		\item Mean file size % adjust block sizes, chunk sizes
		\item Kind of file access % read/write ratio, random reads? random writes? 
		\item Network topology % how much bandwidth do i have 
	\end{itemize}		
\end{frame}


\begin{frame}
	\frametitle{Performance}
	Avoid bottlenecks!\\
	
	Master must not be overloaded.
\end{frame}

\begin{frame}
	\frametitle{Performance in GFS}
	\begin{itemize}
		\item Master stores metadata in RAM.	
		\item Data flow is decoupled from control flow
	\end{itemize}
\end{frame}

\begin{frame}
	\frametitle{Performance in GFS}
	\includegraphics[scale=0.45]{flow.png}
\end{frame}


\section{Scalability}
\subsection*{}
\begin{frame}
	\frametitle{Scalability}
	To scale up in GFS, just add new chunkservers.\\
	Any problem with this?
\end{frame}


\begin{frame}
	\frametitle{ Scalability - GFS }
	Problem: Master stores all metadata in memory. 
\end{frame}

\begin{frame}
	\frametitle{ Scalability - GFS }
	Namespace data is saved using prefix compression. \\
	Master needs only 64 bytes of metadata for every chunk. \\
	$\Rightarrow$ No limitation in practice.
\end{frame}

\section{Availability}
\subsection*{}
\begin{frame}
	\frametitle{Availability}
	Maximize the time all files are accessable.

\end{frame}

\begin{frame}
	\frametitle{Availability}
	\begin{itemize}
		\item Replication
		\item Stale Replica detection 
		\item Replica placement 
		\item Fast Recovery
		\item Master Replication
	\end{itemize}
\end{frame}


\section{Conclusion}
\subsection*{}

\begin{frame}
\frametitle{Conclusion}
	\begin{itemize}
		\item Not trivial
		\item Observe your environment
		\item Choose or develope a filesystem fitting to your environment
		\item Adjust parameters
	\end{itemize}
\end{frame}

\begin{frame}
%too lazy for bibtex
	\frametitle{References}
	$[1]$ Galvin Silberschatz (1994). Operating System concepts, chapter 17 Distributed file systems. Addison-Wesley Publishing Company. \\ 
	$[2]$ Sanjay Ghemawat, Howard Gobioff and Shun-Tak Leung (2003). The Google File System. \url{http://labs.google.com/papers/gfs-sosp2003.pdf}
\end{frame}

\begin{frame}
	Thank you for your attention.
\end{frame}

\end{document}

